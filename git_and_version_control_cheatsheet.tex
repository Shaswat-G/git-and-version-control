\documentclass[a4paper,10pt]{article}
\usepackage[margin=0.8in]{geometry}
\usepackage{xcolor}
\usepackage{titlesec}
\usepackage{fancyhdr}
\usepackage{listings}
\usepackage{hyperref}
\usepackage{courier}

\hypersetup{
    colorlinks=true,
    linkcolor=blue,
    filecolor=magenta,      
    urlcolor=blue,
    pdftitle={Git Command Cheat Sheet},
    pdfpagemode=FullScreen,
}

\titleformat{\section}{\large\bfseries}{\thesection}{1em}{}
\titleformat{\subsection}{\normalsize\bfseries}{\thesubsection}{1em}{}

\pagestyle{fancy}
\fancyhf{}
\rhead{Git Cheat Sheet}
\lhead{Shazz's Reference}
\rfoot{\thepage}

\lstset{
    basicstyle=\footnotesize\ttfamily,
    breaklines=true,
    backgroundcolor=\color{gray!10},
    frame=single,
    tabsize=2
}

\begin{document}

\section*{Git Cheat Sheet}

\section{Git \& Version Control Essentials}
\textbf{Why VCS?} \\
Track file changes over time, collaborate seamlessly, and maintain a complete history of your project.\\[5pt]
\textbf{Types of VCS:}
\begin{itemize}
  \item \textbf{Centralized:} (e.g., SVN, TFS) – Single point of failure, requires constant server access.
  \item \textbf{Distributed:} (e.g., Git, Mercurial) – Every user holds a full copy; enables offline work.
\end{itemize}

\textbf{Tooling Tips:} \\
\begin{itemize}
  \item \textbf{CLI:} Offers full control and conceptual clarity.
  \item \textbf{GUIs \& IDE Extensions:} GitKraken, Sourcetree, VS Code (GitLens) – handy for visual diffing but may hide details.
\end{itemize}

\section{Setup \& Configuration}
\textbf{Installing Git:}
\begin{lstlisting}
# Download from: https://git-scm.com/downloads
git --version  % Verify installation
\end{lstlisting}

\textbf{Initial Configuration:}
\begin{lstlisting}
git config --global user.name "your_name"
git config --global user.email "your_email@example.com"
git config --global core.editor "code --wait"  % Set VS Code as default editor
git config --global -e                           % Edit global config
\end{lstlisting}

\textbf{Managing Line Endings:}
\begin{lstlisting}
# Windows users:
git config --global core.autocrlf true
# Mac/Linux users (if needed):
git config --global core.autocrlf input
\end{lstlisting}

\textbf{Documentation:} \url{https://git-scm.com/docs}

\section{Basic Workflow \& Snapshots}
\textbf{Initializing \& Creating Snapshots:}
\begin{lstlisting}
git init                       % Start a new repository
git add <file> or git add .    % Stage changes
git commit -m "message"        % Commit staged changes
\end{lstlisting}

\textbf{Common Commands:}
\begin{lstlisting}
git status                     % Check working/staging status
git diff                       % View unstaged changes
git diff --staged              % View staged changes
git log --oneline              % Compact commit history
\end{lstlisting}

\section{Reviewing Changes \& History}
\textbf{History & Inspection:}
\begin{lstlisting}
git log                      % Full commit history
git log --oneline --graph    % Visual commit tree
git show <commit_hash>       % Details of a specific commit
git blame <file>             % Identify who last changed each line
\end{lstlisting}

\textbf{Search Options:}
\begin{lstlisting}
git log --grep="term"          % Search commit messages
git log -S"code"               % Find commits modifying specific code
\end{lstlisting}

\section{Branching, Merging \& Advanced Operations}
\subsection{Branching and Stashing}
\begin{lstlisting}
git branch <branch_name>       % Create branch
git switch -c <branch_name>    % Create and switch to branch
git branch -m old new          % Rename branch
git branch -d <branch_name>    % Delete branch

git stash push -m "message"    % Stash changes
git stash list                 % List stashes
git stash pop                  % Apply stash
\end{lstlisting}

\subsection{Merging \& Rebasing}
\begin{lstlisting}
git merge <branch>             % Merge branch into current
git merge --squash <branch>    % Squash merge (combine commits)
git merge --abort              % Abort merge in progress

git rebase master              % Rebase current branch onto master
git rebase --continue          % Continue after conflict resolution
git rebase --abort             % Cancel rebase
\end{lstlisting}

\subsection{Advanced Commands}
\begin{lstlisting}
git commit --amend           % Amend last commit
git cherry-pick <commit_hash>  % Apply specific commit from another branch
git bisect start             % Begin binary search for faulty commit
git bisect good/bad <commit>   % Mark commits during bisect
git bisect reset             % End bisect mode
git tag v1.0                 % Tag the current commit
\end{lstlisting}

\section{Collaboration \& Remotes}
\textbf{Cloning \& Remote Setup:}
\begin{lstlisting}
git clone <url>               % Clone a remote repository
git remote -v                 % List remotes
git remote add upstream <url> % Add new remote (e.g., original repo)
\end{lstlisting}

\textbf{Synchronizing Changes:}
\begin{lstlisting}
git fetch origin              % Fetch remote updates without merging
git pull                      % Fetch and merge remote changes
git push                      % Push local commits to remote
git push origin <branch>      % Push specific branch
git push --set-upstream origin <branch>  % Set upstream for new branch
\end{lstlisting}

\textbf{Tags \& Cleanup:}
\begin{lstlisting}
git push origin <tag>        % Push a tag to remote
git push origin --delete <branch>  % Delete remote branch
git remote prune origin      % Clean up stale remote branches
\end{lstlisting}

\section{Best Practices \& Tips}
\begin{itemize}
  \item \textbf{Commit Often:} Keep commits self-contained and logical.
  \item \textbf{Meaningful Messages:} Use clear, concise commit messages.
  \item \textbf{Review Before Merging:} Use diff tools (e.g., VS Code, KDiff3) to inspect changes.
  \item \textbf{Backup with Branches:} Use feature branches and stashing to avoid losing work.
  \item \textbf{CLI First:} Master the command line before relying solely on GUIs.
\end{itemize}

\textbf{Issue Tracking:} \\
Use GitHub/GitLab Issues and Milestones for bug tracking and feature requests.

\end{document}